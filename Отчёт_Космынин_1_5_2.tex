\documentclass[a4paper,12pt,titlepage,finall]{article}

\usepackage[russian]{babel}
\usepackage{geometry}
\usepackage{multirow}
\usepackage{indentfirst}
\usepackage{pgfplots}
\usepackage{tikz}
\usepackage[final]{listings}

\pgfplotsset{compat=1.18, width=17cm}
\usepgfplotslibrary{fillbetween}

\geometry{a4paper,left=30mm,top=30mm,bottom=30mm,right=30mm}

\setcounter{secnumdepth}{0}

\begin{document}
\lstset{basicstyle=\ttfamily,breaklines=true, numbers=left}

% Титульный лист
\begin{titlepage}
    \begin{center}
    {\small \sc Московский государственный университет \\имени М.~В.~Ломоносова\\ Факультет вычислительной математики и кибернетики\\}
    \vfill
    {\large \sc Отчёт по заданию №6}\\~\\
    {\large \bf <<Сборка многомодульных программ. Вычисление корней уравнений и неопределённых интегралов.>>}\\~\\
    {\large \bf Вариант 1 / 5 / 2}
    \end{center}

    \begin{flushright}
    \vfill
    {Выполнил:\\студент 105 группы\\Космынин~И.~Н.\\~\\
    Преподаватели:\\Гуляев~А.~В.\\Русол~А.~В.}
    \end{flushright}

    \begin{center}
    \vfill
    {\small Москва\\2024}
    \end{center}
\end{titlepage}

\tableofcontents
\newpage

\section{Постановка задачи}

Реализовать численный метод, позволяющий вычислить площадь плоской фигуры, ограниченной тремя кривыми, заданными функциями $f_1(x)=2^x+1$, $f_2(x)=x^5$, $f_3(x)=\frac{1-x}{3}$. Для решения этой задачи используется метод трапеций при вычислении определённого интграла и реализуются два метода для приближённого вычисления абсцисс вершин фигуры (корней уравнений): метод хорд и метод деления отрезка пополам. При этом необходимо аналитически определить отрезки, на которых производится поиск корней.\\

Программа должна иметь модуль написанный на языке Ассемблера, вычисляющий значение функций $f_1$, $f_2$ и $f_3$ в вещественных числах при помощи сопроцессора x87. Соответствующие функции реализуются с соглашением \texttt{cdecl} для дальнейшего использования в Си-коде.

\newpage

\section{Математическое обоснование}

Для анализа предложенных функций построен график:

\begin{figure}[h]
\centering
\begin{tikzpicture}
\begin{axis}[name=graph, axis lines=middle, xlabel=x, ylabel=y, xmin=-4, xmax=4, ymin=-2, ymax=5]
\addplot[color=red,samples=256,thick]{2^x+1};
\addlegendentry{\(y=2^x+1\)}
\addplot[color=green, domain=-2:2, samples=256, thick]{x^5};
\addlegendentry{\(y=x^5\)}
\addplot[color=blue, samples=256, thick]{(1 - x)/3};
\addlegendentry{$y=\frac{1-x}{3}$}
\end{axis}
\end{tikzpicture}
\caption{Плоская фигура, ограниченная графикам заданных функций}
\label{plot1}
\end{figure}

Для корректного нахождения единственного корня уравнения $F(x)=0$ на заданном отрезке $[a,b]$ методами деления отрезка пополам и хорд необходимо, чтобы функция $F$ удовлетворяла на отрезке следующим свойствам$^\cite{math}$:
\begin{enumerate}
    \item $F(a)F(b) < 0$;
    \item $F$ непрерывно дифференцируема на $[a,b]$;
    \item $F'$ монотонна и сохраняет знак на $[a,b]$.
\end{enumerate}

Исходя из этих требований и из графика (рис.~\ref{plot1}) разумно выбрать следующие отрезки для отыскания корней:
\begin{itemize}
\item $F(x) = f_1(x) - f_2(x): [1, 2]$ (т.к. $F'(x) = 2^x\ln{2} - 5x^4$ убывает и $>0$);
\item $F(x) = f_1(x) - f_3(x): [-3, -2]$ (т.к. $F'(x) = 2^x\ln{2} - \frac{1}{3}$ возрастает и $<0$);
\item $F(x) = f_2(x) - f_3(x): [0.6, 1]$ (т.к. $F'(x) = 5x^4 - \frac{1}{3}$ возрастает и $>0$).
\end{itemize}

Далее, для вычисления определённого интеграла методом трапеций на отрезке $[a,b]$ с $n$ отрезками разбияния применяется следующая формула: $\int \limits_a^b f(x) \mathrm{d}x = \frac{b-a}{2n} \left( f(a)+f(b)+2\sum \limits_{k=1}^{n-1}f(x_k) \right) + R$, где $|R| = O(\frac{1}{n^2}) \approx \varepsilon_2$. По правилу Рунге для метода трапеций: $\varepsilon_2 \approx \frac{1}{3}(I_{2n} - I_n)$, где $I_n$ - приближённое значение (без остатка) интеграла на отрезке $[a,b]$ с $n$ отрезками разбиения. Для определения площади потребуется вычислить 3 определённых интеграла, так что требуется увеличение точности в 3 раза. Таким образом, достаточно взять значения $\varepsilon_1 = \varepsilon_2 = 0.0001$ при $\varepsilon = 0.001$ (по условию задачи).

\newpage

\section{Результаты экспериментов}

При помощи программы с точность $\varepsilon_1=0.0001$ найдены координаты вершин фигуры, ограниченной графиками заданных функций (таблица~\ref{table1}).

\begin{table}[h]
\centering
\begin{tabular}{|c|c|c|}
\hline
Кривые & $x$ & $y$ \\
\hline
1 и 2 & 1.2794 & 3.4274 \\
1 и 3 & -2.5222 & 1.1741 \\
2 и 3 & 0.6505 & 0.1165 \\
\hline
\end{tabular}
\caption{Координаты точек пересечения}
\label{table1}
\end{table}

Далее, на рис.~\ref{plot2} закрашена нужная фигура и подписана её площадь, вычисленная также при помощи программы.

\begin{figure}[h]
\centering
\begin{tikzpicture}
\begin{axis}[name=graph, axis lines=middle, xlabel=x, ylabel=y, xmin=-4, xmax=4, ymin=-2, ymax=5, xticklabels={,,}, yticklabels={,,}]
\addplot[color=red, samples=256, thick, name path=A]{2^x+1};
\addlegendentry{\(y=2^x+1\)}

\addplot[color=green, domain=-2:2, samples=256, thick, name path=B]{x^5};
\addlegendentry{\(y=x^5\)}

\addplot[color=blue, samples=256, thick, name path=C]{(1 - x)/3};
\addlegendentry{$y=\frac{1-x}{3}$}

\addlegendimage{empty legend}\addlegendentry{}

\addplot[blue!20, samples=256] fill between[of=A and C, soft clip={domain=-2.5222:0.6505}];
\addplot[blue!20, samples=256] fill between[of=A and B, soft clip={domain=0.6:1.2794}];
\addlegendentry{$S=4.289$}

\addplot[dashed] coordinates { (-2.5222, 1.1741) (-2.5222, 0) };
\addplot[color=black] coordinates { (-2.5222, 0) } node [label={-90:{\small -2.5222}}] {};

\addplot[dashed] coordinates { (1.2794, 3.4274) (1.2794, 0) };
\addplot[color=black] coordinates { (1.2794, 0) } node [label={-90:{\small 1.2794}}] {};

\addplot[dashed] coordinates { (0.6505, 0.1165) (0.6505, 0) };
\addplot[color=black] coordinates { (0.6505, 0) } node [label={-90:{\small 0.6505}}] {};
\end{axis}
\end{tikzpicture}
\caption{Площадь плоской фигуры, ограниченной графикам заданных функций}
\label{plot2}
\end{figure}

\newpage

\section{Структура программы и спецификация функций}

При работе программы используется глобальная константа \texttt{MAX\_STEPS}, задающая максимальное кол-во итераций цикла поиска корня до аварийной остановки. Она нужна, т.к. при неоптимальном выборе отрезка методы поиска корня будут сходиться слишком медленно, из-за чего программа будет нерационально использовать доступные ресурсы. Также в тексте программы определяется глобальная целочисленная переменная \texttt{iter}, в которую сохраняется кол-во итераций, потребовавшееся для последнего вычисления корня.\\

Функции \texttt{f1}, \texttt{f2}, \texttt{f3} для вычисления математических выражений написаны на языке NASM под 32-битную архитектуру и импортируются в Си коде из объектного файла \texttt{funcs.o}.\\

Функция \texttt{main} анализирует параметры командной строки, при необходимости указывает пользователю корректный вариант использования или выводит полную справку. Далее, в зависимости от параметра \texttt{-{}-mode}, возможны 3 варианта работы программы:
\begin{enumerate}
\item вызов функции \texttt{root} для вычисления абсцисс вершин плоской фигуры, вызов функции \texttt{integral} для вычисления определённого интеграла на интересующем отрезке под каждой функцией и вывод значения площади;
\item вызов функции \texttt{root} с пользовательскими параметрами и вывод значения корня или сообщения об ошибке;
\item вызов функции \texttt{integral} с пользовательскими параметрами и вывод значения определённого интеграла.
\end{enumerate}

Полный список используемых функций:
\begin{itemize}
\item \texttt{double root(double (*f)(double), double (*g)(double), double a, double b, double eps1)} вычисляет корень уравнения $f(x) - g(x) = 0$ на отрезке $[a,b]$ с точностью \texttt{eps1} методом хорд или деления отрезка пополам (выбор метода осуществляется на этапе компиляции);
\item \texttt{double integral(double (*f)(double), double a, double b, double eps2)} вычисляет $\int \limits_a^b f(x) \mathrm{d}x$ с точность \texttt{eps2} методом трапеций;
\item \texttt{double area(int flag\_inter, flag\_iter, double eps)} вычисляет площадь фигуры, ограниченной кривыми, заданными функциями $f_1$, $f_2$, $f_3$, с точность \texttt{eps}. Флаги \texttt{flag\_inter} и \texttt{flag\_iter} показывают, нужно ли выводить абсциссы вершин фигуры и кол-во итераций, потребовавшихся для их нахождения, соответственно;
\item \texttt{void print\_usage(char *arg)} выводит справку о вариантах использования программы, параметр \texttt{arg} отвечает за название исполняемого файла;
\item \texttt{void print\_help(char *arg)} выводит полную справку по программе, параметр \texttt{arg} отвечает за название исполняемого файла.
\end{itemize}

\newpage

\section{Сборка программы (Make-файл)}

Makefile содержит следующие основные цели:
\begin{enumerate}
\item \texttt{all} - сборка программы;
\item \texttt{clean} - удаление всех объектных и исполняемого файлов;
\item \texttt{arc} - создания zip-архива со всеми файлами проекта.
\end{enumerate}

По заданию требуется реализовать методы хорд и деления отрезка пополам для поиска корня уравнения на заданном отрезке. Выбор метода решения определяется на этапе компиляции при помощи флага \texttt{-D}. Makefile предоставляет пользователю возможность выбора при помощи переменной \texttt{method}. При значении \texttt{SECANTS} используется метод хорд, а при значении \texttt{SEGMENTS} --- метод деления отрезка пополам (он же применяется и при неправильном значении переменной \texttt{method}).\\

\begin{lstlisting}[language=make, caption={Текст Makefile}, captionpos=b]
.PHONY all clean arc

all: main.o funcs.o main_prog

funcs.o:
    nasm -f elf32 funcs.asm -o funcs.o

main.o:
    gcc -m32 -c main.c -o main.o -D $(method)

main_prog:
    gcc -m32 -lm main.o funcs.o -o main_prog

clean:
    rm *.o main_prog

arc:
    mkdir -p ARC
    zip `date +%Y.%m.%d_%N`.zip main_prog main.c Makefile funcs.asm
    mv *.zip ARC/
\end{lstlisting}

\newpage

\section{Отладка программы, тестирование функций}

Для тестирования и отладки основного модуля на языке Си заданы специальные режимы работы программы (только решение уравнения или только вычисление интеграла), которые задаются при помощи параметра \texttt{-{}-mode}. В ходе тестирования полученные программой значения сравнивались с ответами Wolfram Mathematica.\\

Отладка функций, написанных на языке Ассемблера производилась при помощи инструментов SASM. В частности, велось наблюдение за значениями регистров сопроцессора x87.

\newpage

\section{Программа на Си и на Ассемблере}

Полный код программы, включащий в себя модули на языках Си и Ассембелер, а также Makefile, приложен к отчёту и находится в том же архиве. Си код содержится в файле \texttt{main.c}, реализация математических функций на языке Ассемблера --- в файле \texttt{funcs.asm}.

\newpage

\section{Анализ допущенных ошибок}

При написании модуля на языке Ассемблера сначала были допущены ошибки, связанные с вычислением $2^x$, а также с тем, что возвращаемое значение сохранялось в регистре \texttt{eax}, а не в \texttt{st0}.\\

Кроме того, изначально функция \texttt{root} не учитывала характер выпуклости функции $\varphi = f - g$, из-за чего могла уйти в бесконечный цикл.

\newpage

\begin{raggedright}
\addcontentsline{toc}{section}{Список цитируемой литературы}
\begin{thebibliography}{99}
\bibitem{math} Ильин~В.\,А., Садовничий~В.\,А., Сендов~Бл.\,Х. Математический анализ. Т.\,1~---
    Москва: Наука, 1985.
\end{thebibliography}
\end{raggedright}

\end{document}